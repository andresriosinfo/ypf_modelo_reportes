\documentclass[11pt]{article}

% ================== PAQUETES BÁSICOS ==================
\usepackage[a4paper,margin=2cm]{geometry}
\usepackage{graphicx}
\usepackage{xcolor}
\usepackage{fontspec}   % XeLaTeX / LuaLaTeX
\usepackage{titlesec}
\usepackage{fancyhdr}
\usepackage{array}
\usepackage{tabularx}
\usepackage{lastpage}

% ================== FUENTES (VERSIÓN QUE SIEMPRE COMPILA) ==================
% Para que compile sin depender de tus .ttf, uso fuentes TeX Gyre,
% que están disponibles en Overleaf por defecto.
% Luego puedes cambiar estas líneas por tus archivos de fuente reales.

% Títulos (simulando SE Optimist)
\newfontfamily\SEOptimist{TeX Gyre Adventor}
\newfontfamily\SEOptimistLight{TeX Gyre Adventor}

% Cuerpo (simulando Helvetica)
\newfontfamily\SEBodyMedium{TeX Gyre Heros}
\newfontfamily\SEBodyLight{TeX Gyre Heros}

% Fuente por defecto del cuerpo
\setmainfont{TeX Gyre Heros}

% Alineación del cuerpo según requerimiento
\renewcommand{\familydefault}{\sfdefault}

% ================== COLORES SCHNEIDER ==================
\definecolor{SE355C}{RGB}{0,149,63}       % Pantone 355C (verde principal)
\definecolor{SE369C}{RGB}{120,190,32}     % Pantone 369C
\definecolor{SE375C}{RGB}{141,199,63}     % Pantone 375C
\definecolor{SECoolGrey11}{RGB}{83,86,90} % Cool Grey 11
\definecolor{SECoolGrey7}{RGB}{140,144,147} % Cool Grey 7

% ================== FORMATO DEL PÁRRAFO ==================
\setlength{\parindent}{0pt}     % sin sangría
\setlength{\parskip}{0.8em}     % espacio entre párrafos
\raggedright                    % alineado a la izquierda

% ================== ESTILO DE TÍTULOS ==================
% Título de nivel 1 = SE Optimist, verde principal
\titleformat{\section}
  {\SEOptimist\Large\color{SE355C}}
  {\thesection.}{0.6em}{}

% Subtítulo = SE Optimist Light
\titleformat{\subsection}
  {\SEOptimistLight\large\color{SE355C}}
  {\thesubsection}{0.6em}{}

% Sub-subtítulo = Helvetica Medium
\titleformat{\subsubsection}
  {\SEBodyMedium\normalsize\color{SECoolGrey11}}
  {\thesubsubsection}{0.6em}{}

% Reducir espacio después de títulos
\titlespacing*{\section}{0pt}{1.5ex}{1ex}
\titlespacing*{\subsection}{0pt}{1.2ex}{0.7ex}
\titlespacing*{\subsubsection}{0pt}{1ex}{0.5ex}

% ================== ENCABEZADO Y PIE DE PÁGINA ==================
\pagestyle{fancy}
\fancyhf{} % limpiar

% Encabezado: logo a la izquierda, nombre del sistema a la mitad, fecha a la derecha
% Cuando tengas el logo, quita el % y asegúrate que el archivo exista
% \lhead{%
%   \includegraphics[height=12mm]{schneider_logo} % <-- schneider_logo.pdf/png en la misma carpeta
% }
\chead{%
  \SEOptimist\color{SECoolGrey11} YPF Argentina
}
\rhead{%
  \SEBodyLight\small\color{SECoolGrey11}
  {{FECHA\_REPORTE}} % <-- luego Cursor reemplaza esto
}

% Línea verde bajo el encabezado
\renewcommand{\headrule}{%
  {\color{SE355C}\hrule height 1.2pt width\headwidth\vspace{-0.5em}}%
}

% Pie de página: nombre reporte izquierda, número de página derecha
\lfoot{\SEBodyLight\small YPF Argentina}
\rfoot{\SEBodyLight\small Página \thepage\ de \pageref{LastPage}}

\renewcommand{\footrulewidth}{0pt}

% ================== METADATOS DEL DOCUMENTO ==================
\title{\SEOptimist\color{SE355C} Informe de Turno de Operación}
\author{\SEBodyLight Schneider Electric}
\date{} % Fecha la ponemos en el encabezado

% ============================================================
% ===============   INICIO DEL DOCUMENTO   ====================
% ============================================================
\begin{document}

% --------- PORTADA / RESUMEN VISUAL (PÁGINA 1) --------------
\thispagestyle{fancy}

\vspace*{2cm}

{\SEOptimist\fontsize{24}{26}\selectfont\color{SE355C}
  INFORME DE TURNO -- UNIDAD N-101}
}

\vspace{1em}

{\SEBodyLight\large\color{SECoolGrey11}
Período analizado: {{PERIODO\_ANALIZADO}} % p.ej. "Últimas 8 horas (07:00–15:00)"
}

\vspace{2em}

% --- Cuadro resumen de indicadores principales ---
\begin{tabularx}{\textwidth}{>{\SEBodyMedium\color{SECoolGrey11}}l
                            >{\SEBodyLight\color{SECoolGrey11}}X}
Indicador & Valor \\
\hline
Tiempo en rango global & {{PCT\_EN\_RANGO\_GLOBAL}} \\
Variables analizadas & {{N\_VARIABLES}} \\
Variables sin desviaciones & {{N\_OK}} \\
Variables con desviaciones & {{N\_CON\_DESVIO}} \\
Variables críticas & {{N\_CRITICAS}} \\
\end{tabularx}

\vspace{2em}

{\SEBodyLight\color{SECoolGrey11}
Este reporte presenta el estado actual del desempeño de la unidad
{{NOMBRE\_UNIDAD}}, incluyendo estabilidad de variables de proceso,
desviaciones relevantes y recomendaciones para el siguiente turno.
}

\newpage

% ================== CONTENIDO PRINCIPAL ==================

\section{Resumen Ejecutivo}

{{RESUMEN\_EJECUTIVO}}

% --------------------------------------------------------

\section{Inventario y Alcance del Análisis}

{{INVENTARIO\_Y\_ALCANCE}}

% --------------------------------------------------------

\section{Estado de las Variables del Turno}

\subsection{Variables bien comportadas}

{{VARIABLES\_BIEN\_COMPORTADAS}}

\subsection{Variables con desviaciones relevantes}

{{ESTADO\_GLOBAL\_DESVIOS}}

\subsection{Detalle por variable con desviaciones}

{{DETALLE\_POR\_VARIABLE}}

% --------------------------------------------------------

\section{Condiciones de Entrada y Salida de la Unidad}

{{CONDICIONES\_ENTRADA\_SALIDA}}

% --------------------------------------------------------

\section{Puntos a Vigilar para el Siguiente Turno}

{{PUNTOS\_A\_VIGILAR}}

% --------------------------------------------------------

\section*{Anexos (Opcional)}

{{ANEXOS\_OPCIONALES}}

\label{LastPage}
\end{document}
